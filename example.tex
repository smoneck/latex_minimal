% This is a comment
\documentclass[
  page=a4,
  DIV=12
  ]{scrartcl}

\usepackage[utf8]{inputenc}
\usepackage[T1]{fontenc}
\usepackage[english]{babel}
\usepackage{amsmath, amssymb}

\title{A simple Testdocument}
\subtitle{LaTeX in short}
\author{Jane Doe}
\date{\today}
\setcounter{tocdepth}{3}

\begin{document}

\maketitle
\tableofcontents

\section{First section}



Hier kommt die Einleitung. Ihre Überschrift kommt
automatisch in das Inhaltsverzeichnis.

\subsection{First subsection}

\LaTeX{} ist auch ohne Formeln sehr nützlich und
einfach zu verwenden. Grafiken, Tabellen,
Querverweise aller Art, Literatur- und
Stichwortverzeichnis sind kein Problem.

Formeln sind etwas schwieriger, dennoch hier ein
einfaches Beispiel. Zwei von Einsteins
berühmtesten Formeln lauten:
%
\begin{align}
E &= mc^2                 \\
m &= \frac{m_0}{\sqrt{1-\frac{v^2}{c^2}}}
\end{align}
%
Aber wer keine Formeln schreibt, braucht sich
damit auch nicht zu beschäftigen.

\end{document}
